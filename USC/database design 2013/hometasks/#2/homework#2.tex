\documentclass[a4paper,11pt]{article}
\usepackage[utf8]{inputenc}
\usepackage[english]{babel}
\usepackage{subfigure}
\usepackage{fullpage}
\usepackage{amsmath, amsthm, amsfonts, amssymb}
\usepackage{cite}
\usepackage{tikz}

\newcommand{\union}[2]{#1 \bigcup #2}
\newcommand{\intersection}[2]{#1 \bigcap #2}
\newcommand{\difference}[2]{#1 \setminus #2}
\newcommand{\selection}[2]{\sigma_{#1}\left(#2\right)}
\newcommand{\projection}[2]{\pi_{#1}\left(#2\right)}
\newcommand{\carProd}[2]{#1 \times #2}
\newcommand{\natJoin}[2]{#1 \Join #2}
\newcommand{\thJoin}[3]{#1 \Join_{#3} #2}
\newcommand{\rename}[3]{\rho_{#1(#2)}\left(#3\right)}



\author{Sergey Aganezov}
\title{Homework assignment $\#2$ \\ CSE 520 ``Database Management Systems'' \\ Fall 2013}
\date{}

\begin{document}
	\maketitle
	\begin{enumerate}
		\item Exercise 2.4.3: d, e, h (pages 55-57) from Ullman and Widom: A First Course in Database Systems


		\textbf{d}: The treaty of Washington in 1921 prohibited capital ships heavier than 35000 tons. List the ships violated the treaty of Washington.

		\begin{equation*}
			\projection{name}{
				\selection{country=USA\ \mathbf{AND}\ displacement > 35000}{
					\natJoin{Ships}{Classes}
				}
			}
		\end{equation*}

		\textbf{e}: List the name, displacement, and number of guns for the ships engadged in the battle of Guadacanal.

		\begin{align*}
			draft\_r\_Battles &:= \rename{Battles}{battle\_name,\ date}{Battles} \\
			r\_Battles &:= \selection{battle\_name=Guadalcanal}{draft\_r\_Battles} \\
			r\_Outcomes &:= \rename{Outcomes}{ship\_name,\ battle\_name,\ result}{Outcomes} \\
			r\_Ships &:= \rename{Ships}{ship\_name,\ class,\ launched}{Ships} \\
			r\_BO &:= \natJoin{r\_Batles}{r\_Outcomes} \\
			r\_BOS &:= \natJoin{r\_BO}{r\_Ships} \\
			draft\_result &:= \natJoin{r\_BOS}{Classes} \\
			result &:= \projection{ship\_name,\ displacement,\ numGuns}{draft\_result}
		\end{align*}

		\textbf{h}: Find those countries that had both battleships and battlecrusiers.

		\begin{align*}
			Classes\_temp &:= \projection{type,\ country}{Classes} \\
			Classes\_1 &:= \rename{Classes\_temp}{type\_1,\ country\_1}{Classes\_temp} \\
			Classes\_2 &:= \rename{Classes\_temp}{type\_2,\ country\_2}{Classes\_temp} \\
			draf\_result &:= \thJoin{Classes\_1}{Classes\_2}{country\_1= country\_2\ \mathbf{AND}\ type\_1 \neq type\_2} \\
			result &:= \projection{country\_1}{draft\_result} \\
			%  \thJoin{}{}{  } \\ 
		\end{align*}
		% \begin{align*}
			
			% 
			% 
			% 
		% \end{align*}

		\pagebreak

		\item Exercise 2.5.2: a, b, c (page 63) from Ullman and Widom: A First Course in Database Systems

		\textbf{a}: No class of ships may have guns with larger than 16-inch bore.

		\begin{equation*}
			\selection{bore > 16}{Classes} = \emptyset
		\end{equation*}

		\textbf{b}: If a class of ships has more than 9 guns, then their bore must be no larger than 14 inches.

		\begin{equation*}
			\selection{numGums > 9\ \mathbf{AND}\ bore > 14}{Classes} = \emptyset
		\end{equation*}

		\textbf{c}: No class my have more than 2 ships

		\begin{align*}
			table &:= \projection{class,\ name}{Ships} \\
			table\_1 &:= \rename{table}{class\_1,\ name\_1}{table} \\
			table\_2 &:= \rename{table}{class\_2,\ name\_2}{table} \\
			table\_3 &:= \rename{table}{class\_3, name\_3}{table} \\
			f\_result &:= \thJoin{table\_1}{table\_2}{class\_1 = class\_2\ \mathbf{AND}\ name\_1\neq name\_2} \\
			result &:= \thJoin{f\_result}{table\_3}{class\_1=class\_3\ \mathbf{AND}\ name\_1\neq name\_3\ \mathbf{AND}\ name\_2\neq name\_3} \\
			result &:= \emptyset
		\end{align*}
	\end{enumerate}
\end{document}